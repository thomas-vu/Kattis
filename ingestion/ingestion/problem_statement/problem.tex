\problemname{A Question of Ingestion}
Stan Ford is a typical college graduate student, meaning that one of the most
important things on his mind is where his next meal will be.  Fortune has smiled
on him as he's been invited to a multi-course barbecue put on by some of the
corporate sponsors of his research team, where each course lasts exactly one
hour.  Stan is a bit of an analytical type and has determined that his eating
pattern over a set of consecutive hours is always very consistent.  In the first
hour, he can eat up to $m$ calories (where $m$ depends on factors such as
stress, bio-rhythms, position of the planets, etc.), but that amount goes down by
a factor of two-thirds each consecutive hour afterwards (always truncating in
cases of fractions of a calorie). However, if he stops eating for one hour,
the next hour he can eat at the same rate as he did before he stopped.  So,
for example, if $m=900$ and he ate for five consecutive hours, the most he could
eat each of those hours would be $900$, $600$, $400$, $266$ and $177$ calories,
respectively. If, however, he didn't eat in the third hour, he could then eat
$900$, $600$, $0$, $600$ and $400$ calories in each of those hours.  Furthermore, if Stan
can refrain from eating for two hours, then the hour after that he's capable of
eating $m$ calories again.  In the example above, if Stan didn't eat during
the third and fourth hours, then he could consume $900$, $600$, $0$, $0$ and $900$
calories.

Stan is waiting to hear what will be served each hour of the barbecue as he
realizes that the menu will determine when and how often he should refrain from
eating.  For example, if the barbecue lasts $5$ hours and the courses served
each hour have calories $800$, $700$, $400$, $300$, $200$ then the best strategy when
$m=900$ is to eat every hour for a total consumption of $800+600+400+266+177 =
2\,243$ calories. If however, the third course is reduced from $400$ calories to $40$
calories (some low-calorie celery dish), then the best strategy is to not eat
during the third hour --- this results in a total consumption of $1\,900$ calories.

The prospect of all this upcoming food has got Stan so frazzled he can't think
straight.  Given the number of courses and the number of calories for each
course, can you determine the maximum amount of calories Stan can eat?

\section*{Input}
Input starts with a line containing two positive integers $n$ $m$
($n \leq 100, m \leq 20\,000$) indicating the number of courses and the number of
calories Stan can eat in the first hour, respectively.  The next line contains
$n$ positive integers indicating the number of calories for each course.

\section*{Output}
Display the maximum number of calories Stan can consume.
